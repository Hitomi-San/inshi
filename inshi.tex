%! LuaLaTeX 文書
\documentclass[report]{dennou777}

\usepackage[unicode,colorlinks=true]{hyperref}
\hypersetup{linkcolor=blue,urlcolor=teal}

\usepackage{luatexja}
\ltjsetparameter{jacharrange={-2,-3,-8}}
\usepackage[no-math,match,deluxe]{luatexja-preset}

\usepackage{textcomp}
\usepackage{luatexja-otf}

\usepackage{graphicx,xcolor}
\usepackage{pxrubrica}
\usepackage{tcolorbox}
\usepackage{indentfirst}
\usepackage{bxwareki}
\usepackage{array}

\usepackage{tikz}
\usetikzlibrary{plotmarks}

\usepackage[T1]{fontenc}
\usepackage{amsmath,mathtools,amssymb,mathrsfs,rsfso,mleftright}
\usepackage[math]{kurier}
\usepackage[euler-digits]{eulervm}
\usepackage[scaled]{beramono}
\allowdisplaybreaks[4]

\usepackage[a4paper]{geometry}

%%%%%%%%%% フォントの設定
\setsansfont[Ligatures=TeX,]{nishiki-teki}
\setsansjfont[Ligatures=TeX,]{nishiki-teki}
\setmainfont[Ligatures=TeX,]{nishiki-teki}
\setmainjfont[Ligatures=TeX,]{nishiki-teki}

%%%%%%%%%% 以下は自前コマンド

\Dtitle{北海道大学理学院 入試問題 解答例}
\Dauthor{ひとみさん}
\date{\warekitoday}

\begin{document}
\maketitle

\chapter{平成 31 年度}
\section{問題 I}

\chapter{平成 30 年度}
\section{問題 I}
以下の問 1 から問 2 までの全ての設問に答えよ。

\subsection*{問 1}
質量 $m$ の質点が、原点 $O$ から距離 $r$ の位置にあるとき、そのポテンシャル
エネルギーが
\[ U(r)=-\frac{C}{\alpha-1}\frac{m}{r^{\alpha-1}}\]
で与えられる中心力場を考える。ここで、 $C$ と $\alpha$ は $C>0, \alpha>1$ を
満たす定数である。このとき、以下のようにして質点の軌道を求めよ。

\begin{itemize}
	\item[1-1.] この質点が運動する平面内で、上図のような座標系を
		考える。質点の位置ベクトルの大きさを $r$、位置ベクトルが
		$x$ 軸となす角を $\theta$ とする。質点のラグランジアン
		$\mathcal{L}=\mathcal{L}(r,\theta,\dot{r},\dot{\theta})$
		を書き下せ。ただし $\dot{r}=dr/dt, \dot{\theta}=d\theta/dt$
		とする。
	\item[1-2.] 上で求めたラグランジアンから、$r$ と $\theta$ に関する
		運動方程式を求めよ。また、角運動量 $L=mr^2\dot{\theta}$ が
		保存することを示せ。
	\item[1-3.] $L\neq0$ の場合には、単位質量あたりの角運動量 $l=L/m$ を
		用いて、時間微分を $\theta$ に関する微分に変換することができる。
		その結果、$r$ に関する運動方程式は
		\begin{equation}
			\frac{d^2}{d\theta^2}\mleft(\frac{1}{r}\mright)+
			\mleft(1-r^{3-\alpha}\frac{C}{l^2}\mright)\frac{1}{r}=0
			\label{30-1-1-3}
		\end{equation}
		と書けることを示せ。
	\item[1-4.] $\alpha=3$の場合に、式\eqref{30-1-1-3}の一般解を、$C/l^2$ の
		値に応じて 3 つの場合に分けて求めよ。
\end{itemize}

\subsection*{解答 1}
\begin{itemize}
	\item[1-1.]
		\begin{align*}
			\mathcal{L}&=T-U\qquad\text{($T$ は運動エネルギー)}\\
			&=\frac{1}{2}m\mathbold{v}^2-U(r)\\
			&=\frac{1}{2}m\mleft[\dot{r}^2-(r\dot{\theta})^2\mright]+
				\frac{C}{\alpha-1}\frac{m}{r^{\alpha-1}}.\\
		\end{align*}
	\item[1-2.]
		$r$ についてのラグランジュの運動方程式は、
		\begin{align*}
			\frac{\partial\mathcal{L}}{\partial\dot{r}}
			&=\frac{1}{2}m\mleft(2\dot{r}\mright)\\
			&=m\dot{r},\\
			\frac{\partial\mathcal{L}}{\partial r}
			&=-\frac{1}{2}\mleft[2\dot{\theta}^2r
				+(\alpha-1)\frac{C}{(\alpha-1)r^\alpha}\mright]\\
			&=m\dot{\theta}^2r-\frac{m}{2}\frac{C}{r^\alpha}
		\end{align*}
		であるから、
		\begin{align*}
			\frac{d}{dt}\mleft(\frac{\partial\mathcal{L}}{\partial\dot{r}}\mright)
				-\frac{\partial\mathcal{L}}{\partial r}&=0\\
			m\ddot{r}-m\dot{\theta}^2r+m\frac{C}{r^\alpha}&=0\\
			\ddot{r}+Cr^{-\alpha}-\dot{\theta}^2r&=0.
		\end{align*}

		$\theta$ についてのラグランジュの運動方程式は、
		\begin{align*}
			\frac{\partial\mathcal{L}}{\partial\dot{\theta}}
			&=\frac{1}{2}m(2r^2\dot{\theta})=mr^2\dot{\theta},\\
			\frac{\partial\mathcal{L}}{\partial\theta}&=0
		\end{align*}
		であるから、
		\begin{align*}
			\frac{d}{dt}\mleft(\frac{\partial\mathcal{L}}{\partial\dot{\theta}}\mright)
				-\frac{\partial\mathcal{L}}{\partial\theta}&=0\\
			\frac{d}{dt}[mr^2\dot{\theta}]&=0.
		\end{align*}

		ここで、角運動量 $L=mr^2\dot{\theta}$ を用いると、
		$\theta$ についてのラグランジュの運動方程式は、
		\[\frac{dL}{dt}=0\]
		とかけるので、角運動量は保存する。
	\item[1-3.]
		$l=L/m=r^2\dot{\theta}$ であるから、$d\theta/dt=l/r^2$ である。
		これを用いて、$d^2/dt^2$ を計算する。

		\begin{align*}
			\frac{dr}{dt}&=\frac{dr}{d\theta}\frac{d\theta}{dt}
				=\frac{l}{r^2}\frac{dr}{d\theta}
				=l\frac{d}{d\theta}\frac{1}{r},\\
			\frac{d^2r}{dt^2}&=\frac{d}{dt}\frac{dr}{dt}\\
			&=\frac{d}{dt}\mleft[-l\frac{d}{d\theta}\frac{1}{r}\mright]\\
			&=-l\frac{d}{dt}\frac{d}{d\theta}\frac{1}{r}\\
			&=-l\mleft(\frac{d\theta}{dt}\frac{d}{d\theta}\mright)\frac{d}{d\theta}\frac{1}{r}\\
			&=-l\frac{l}{r^2}\frac{d^2}{d\theta^2}\frac{1}{r}\\
			&=-\frac{l^2}{r^2}\frac{d^2}{d\theta^2}\frac{1}{r}
		\end{align*}

		これを $r$ についてのラグランジュの運動方程式に代入すると、
		\begin{align*}
			\frac{d^2r}{dt^2}+Cr^{-\alpha}-\mleft(\frac{d\theta}{dt}\mright)^2&=0\\
			-\frac{l^2}{r^2}\frac{d^2}{d\theta^2}\frac{1}{r}+Cr^{-\alpha}
				-\mleft(\frac{l}{r^2}\mright)^2r&=0\\
			\frac{d^2}{d\theta^2}\frac{1}{r}+C\frac{r^{2-\alpha}}{l^2}-\frac{1}{r}&=0\\
			\frac{d^2}{d\theta^2}\mleft(\frac{1}{r}\mright)+
				\mleft(1-r^{3-\alpha}\frac{C}{l^2}\mright)\frac{1}{r}=0
		\end{align*}
		となり、式\eqref{30-1-1-3}が得られた。
	\item[1-4.]
		$\alpha=3$ のとき、$C/l^2=\gamma, r^{-1}=R$ とおくと、
		式\eqref{30-1-1-3}は次のようにかける。
		\begin{equation}
			\frac{d^2R}{d\theta^2}-(\gamma-1)R=0\label{30-1-1-4-a}
		\end{equation}

		ここで、二次方程式 $\lambda^2-(\gamma-1)=0$ の解は、$\lambda=\pm\sqrt{\gamma-1}$
		である。

		$\gamma=1$ のとき、$\lambda=0$ となって、\eqref{30-1-1-4-a} の解は、
		$R=c_1\theta+c_2$ となる($c_1, c_2$ は任意定数)。

		$\gamma\neq1$ のとき、\eqref{30-1-1-4-a} の解は、
		$R=c_1\exp[\sqrt{\gamma-1}\,\theta]+c_2\exp[-\sqrt{\gamma-1}\,\theta]$
		となる。

		さらに、$\gamma<1$ のとき、$\sqrt{1-\gamma}=\beta$ とおいて
		さらに式を変形して、
		\begin{align*}
			R&=c_1\exp[\sqrt{\gamma-1}\,\theta]+c_2\exp[-\sqrt{\gamma-1}\,\theta]\\
			&=c_1\exp[i\beta\theta]+c_2\exp[-i\beta\theta]\\
			&=\frac{d_1-id_2}{2}\exp[i\beta\theta]+\frac{d_1+id_2}{2}\exp[-i\beta\theta]
				\qquad\text{($d_1, d_2$ は任意定数)}\\
			&=d_1\frac{e^{i\beta\theta}+e^{-i\beta\theta}}{2}
				-id_2\frac{e^{i\beta\theta}-e^{-i\beta\theta}}{2}\\
			&=d_1\frac{e^{i\beta\theta}+e^{-i\beta\theta}}{2}
				+d_2\frac{e^{i\beta\theta}-e^{-i\beta\theta}}{2i}\\
			&=d_1\cos(\beta\theta)+d_2\sin(\beta\theta)
		\end{align*}

		以上を整理して、一般解は、
		\begin{equation}
			r^{-1}=
			\begin{cases}
				c_1\cos\mleft(\sqrt{1-\dfrac{C}{l^2}}\,\theta\mright)+
					c_2\sin\mleft(\sqrt{1-\dfrac{C}{l^2}}\,\theta\mright)
					&\quad\text{$C/l^2<1$ の場合}\\
				c_1+c_2\theta
					&\quad\text{$C/l^2=1$ の場合}\\
				c_1\exp\mleft(\sqrt{\dfrac{C}{l^2}-1}\,\theta\mright)+
					c_2\exp\mleft(-\sqrt{\dfrac{C}{l^2}-1}\,\theta\mright)
					&\quad\text{$C/l^2>1$ の場合}
			\end{cases}
		\end{equation}
		である。
\end{itemize}

\subsection*{問2}
質量 $m$ の惑星が、質量 $m$ の恒星の周りを運動しており、恒星と惑星の間には
万有引力が働いている。以下では万有引力定数を $G$ とする。また、$m$ は
$M$ に比べて十分に小さく、この系の重心は恒星の位置と同じであるとする。

\begin{itemize}
	\item[2-1.] 恒星が半径 $r$ の等速円運動をしているとき、惑星の運動エネルギー
		$T$ を $M,m,r,G$ を用いて表わせ。
	\item[2-2.] 次に、恒星風などにより恒星の質量が減少する場合に、惑星の軌道が
		どのように変化するかを考える。はじめ恒星の質量は $M$ であり、惑星の
		円軌道の半径は $r$ であった。その後惑星の質量は、惑星の公転周期に
		比べて十分ゆっくりと微小変化し、$M+dM$ になった。その結果、惑星の
		円軌道の半径が $r+dr$ に変化したとする。

		この場合にも、問 1 と同様に中心力だけが働いているため、恒星質量が
		変化する前後で、惑星の角運動量が保存する。この事を用い、2 次以上の
		微小量を無視することで、$dM$ と $dr$ の間に成り立つ関係式を求めよ。
	\item[2-3.] 初期の恒星質量を $M_0$、惑星の軌道半径を $r_0$ とする。恒星の
		質量が、2-2 で求めた関係式が成り立つように、十分ゆっくりと $M_0/2$
		まで減少した。その結果、惑星の軌道半径は $r_0$ の何倍になるか求めよ。
	\item[2-4.] 次に、質量変化が速い場合の極限を考える。恒星の質量が $M_0$ から
		$M_0/2$ に瞬間的に減少したとき、惑星の運動はどのように変化するか
		定性的に説明せよ。
\end{itemize}

\subsection*{解答}
\begin{itemize}
	\item[2-1.]
		惑星に働く万有引力は $F=GmM/r^2$ である。惑星の運動の速さを $v$ と
		すると、円運動をしているので、$F=mv^2/r$ が成り立つ。

		惑星の運動エネルギーは、
		\[T=\frac{1}{2}mv^2=\frac{1}{2}rF=\frac{GmM}{2r}\]
	\item[2-2.]
		はじめの角速度を $\omega$ とすると、
		\begin{align*}
			\omega&=\sqrt{\frac{F}{mr}}\\
			&=\sqrt{\frac{1}{mr}\cdot G\frac{mM}{r^2}}\\
			&=\frac{1}{r}\sqrt{G\frac{M}{r}}.
		\end{align*}
		恒星質量が変化した後の角速度を $\omega'$ とすると、
		\begin{align*}
			\omega'&=\sqrt{\frac{F'}{m(r+dr)}}\\
			&=\sqrt{\frac{1}{m(r+dr)}\cdot G\frac{m(M+dM)}{(r+dr)^2}}\\
			&=\frac{1}{r+dr}\sqrt{G\frac{M+dM}{r+dr}}.
		\end{align*}
	
		角運動量が保存するので、
		\begin{align*}
			mr^2\omega&=m(r+dr)^2\omega'\\
			r^2\frac{1}{r}\sqrt{G\frac{M}{r}}&=(r+dr)^2\frac{1}{r+dr}\sqrt{G\frac{M+dM}{r+dr}}\\
			\sqrt{Mr}&=\sqrt{(M+dM)(r+dr)}\\
			Mr&=(M+dM)(r+dr)\\
			Mr&=Mr+rdM+Mdr\qquad\text{($dM$ と $dr$ の積を無視した)}\\
			\frac{dM}{dr}&=-\frac{M}{r}.
		\end{align*}
	\item[2-3.]
		十分ゆっくりと質量が変化し、つねに $dM/dr=-M/r$ が成り立っているので、
		$dM$ を $\Delta M=-M_0/2$ と置き換えることができる。この場合、
		\begin{align*}
			\frac{dM}{dr}\sim\frac{\Delta M}{\Delta r}&=-\frac{M_0}{r_0}\\
			\frac{-M_0/2}{\Delta r}&=-\frac{M_0}{r_0}\\
			\Delta r&=\frac{r_0}{M_0}\cdot\frac{M_0}{2}=\frac{r_0}{2}
		\end{align*}
		となって、$r$ の変分は $\Delta r=r_0/2$ である。したがって、軌道半径は
		$r_0$ の $3/2$ 倍になる。
	\item[2-4.]
		恒星質量が瞬間的に減少した場合、惑星に働く向心力が減少し、惑星が軌道の
		外側に向かって弾き飛ばされる。
\end{itemize}

\section{問題 II}
以下の問 1 から問 2 までのすべての設問に答えよ。

\subsection*{問 1}
真空中に、半径 $a$ の導体円盤 A, B が間隔 $d$ で配置された平行平板コンデンサー
がある。図 \ref{30-2} に示すように座標軸を取り、導体 B の中心を原点 $O$ とする。
以下では、真空中の誘電率を $\epsilon_0$、透磁率を $\mu_0$ とする。また、
$a\gg d$ であり、導体の端における電場の乱れは無視できるとする。

\begin{itemize}
	\item[1-1.] ガウスの法則を用い、導体間の電場の大きさ $E$ とコンデンサーの
		静電容量 $C$ を求めよ。
	\item[1-2.] コンデンサーに蓄えられた静電エネルギー $U$ は導体 A と B の
		電荷が $0$ から、それぞれ $+Q$ と $-Q$ になるまで導体間で電荷を
		移動させるのに必要な仕事に等しい。コンデンサーに蓄えられた静電
		エネルギーは、$U=Q^2/2C$ となることを示せ。
	\item[1-3.] 導体 A と B がそれぞれ $+Q, -Q$ に帯電しているとき、導体間に
		働く力の大きさを求めよ。
	\item[1-4.] それぞれ $+Q, -Q$ に帯電した導体 A, B 間に、自己インダクタンス
		$L$ のコイルを時刻 $t=0$ に接続した。時刻 $t$ における、導体 A の
		電荷 $Q(t)$ と導体 A から B へ流れる電流 $I(t)$ を求めよ。ただし
		コンデンサーの静電容量を $C$ とし、コイル以外の回路のインダクタンス
		は無視できるものとする。
	\item[1-5.] このときコイルに蓄えられたエネルギー $(1/2)LI^2$ とコンデンサー
		に蓄えられたエネルギー $U$ との和が保存することを示せ。
	\item[1-6.] つぎにコイルを外し、帯電していないコンデンサーに直流電源を接続
		して、交流電圧 $V(t)=V_0\sin\omega t$ を加えた。導体間に生じる変位電流
		を求めよ。ただしコンデンサーの静電容量を $C$ とする。
	\item[1-7.] また、このとき導体 A と導体 B の間の位置 $(x,0,d/2)$ における
		磁束密度 $B$ を求めよ。ただし $0<x<a$ とする。
\end{itemize}
\end{document}
