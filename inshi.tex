%! LuaLaTeX 文書
\documentclass[report]{dennou777}

\usepackage[unicode,colorlinks=true]{hyperref}
\hypersetup{linkcolor=blue,urlcolor=teal}

\usepackage{luatexja}
\ltjsetparameter{jacharrange={-2,-3,-8}}
\usepackage[no-math,match,deluxe]{luatexja-preset}

\usepackage{textcomp}
\usepackage{luatexja-otf}

\usepackage{graphicx,xcolor}
\usepackage{pxrubrica}
\usepackage{tcolorbox}
\usepackage{indentfirst}
\usepackage{bxwareki}
\usepackage{array}

\usepackage{tikz}
\usetikzlibrary{plotmarks}

\usepackage[T1]{fontenc}
\usepackage{amsmath,mathtools,amssymb,mathrsfs,mleftright}
\usepackage[math]{kurier}
\usepackage[euler-digits]{eulervm}
\allowdisplaybreaks[4]

\usepackage[a4paper]{geometry}

%%%%%%%%%% フォントの設定
\setsansfont[Ligatures=TeX,]{nishiki-teki}
\setsansjfont[Ligatures=TeX,]{nishiki-teki}
\setmainfont[Ligatures=TeX,]{nishiki-teki}
\setmainjfont[Ligatures=TeX,]{nishiki-teki}

%%%%%%%%%% 以下は自前コマンド

\Dtitle{北海道大学理学院 入試問題 解答例}
\Dauthor{ひとみさん}
\date{\warekitoday}

\begin{document}
\maketitle

\chapter{平成 31 年度}
\section{問題 I}

\chapter{平成 30 年度}
\section{問題 I}
以下の問 1 から問 2 までの全ての設問に答えよ。
\subsection*{問 1}
質量 $m$ の質点が、原点 $O$ から距離 $r$ の位置にあるとき、そのポテンシャル
エネルギーが
\[ U(r)=-\frac{C}{\alpha-1}\frac{m}{r^{\alpha-1}}\]
で与えられる中心力場を考える。ここで、 $C$ と $\alpha$ は $C>0, \alpha>1$ を
満たす定数である。このとき、以下のようにして質点の軌道を求めよ。

\begin{itemize}
	\item[1-1.] この質点が運動する平面内で、上図のような座標系を
		考える。質点の位置ベクトルの大きさを $r$、位置ベクトルが
		$x$ 軸となす角を $\theta$ とする。質点のラグランジアン
		$\mathscr{L}=\mathscr{L}(r,\theta,\dot{r},\dot{\theta})$
		を書き下せ。ただし $\dot{r}=dr/dt, \dot{\theta}=d\theta/dt$
		とする。
	\item[1-2.] 上で求めたラグランジアンから、$r$ と $\theta$ に関する
		運動方程式を求めよ。また、角運動量 $L=mr^2\dot{\theta}$ が
		保存することを示せ。
	\item[1-3.] $L\neq0$ の場合には、単位質量あたりの角運動量 $l=L/m$ を
		用いて、時間微分を $\theta$ に関する微分に変換することができる。
		その結果、$r$ に関する運動方程式は
		\begin{equation}
			\frac{d^2}{d\theta^2}\mleft(\frac{1}{r}\mright)+
			\mleft(1-r^{3-\alpha}\frac{C}{l^2}\mright)\frac{1}{r}=0
			\label{30-1-1-3}
		\end{equation}
		と書けることを示せ。
	\item[1-4.] $\alpha=3$の場合に、式\eqref{30-1-1-3}の一般解を、$C/l^2$ の
		値に応じて 3 つの場合に分けて求めよ。
\end{itemize}

\subsection*{解答 1}
\begin{itemize}
	\item[1-1.]
		\begin{align*}
			\mathscr{L}&=T-U\qquad\text{($T$ は運動エネルギー)}\\
			&=\frac{1}{2}m\mathbold{v}^2-U(r)\\
			&=\frac{1}{2}m\mleft[\dot{r}^2-(r\dot{\theta})^2\mright]+\frac{C}{\alpha-1}\frac{m}{r^{\alpha-1}}\\
			&=\frac{1}{2}m\mleft[\dot{r}^2-(r\dot{\theta})^2+\frac{C}{(\alpha-1)r^{\alpha-1}}\mright]
		\end{align*}
	\item[1-2.]
		\begin{align*}
			\frac{\partial\mathscr{L}}{\partial\dot{r}}
			&=\frac{1}{2}m\mleft(2\dot{r}\mright)=m\dot{r},\\
			\frac{\partial\mathscr{L}}{\partial r}
			&=-\frac{1}{2}\mleft[2\dot{\theta}^2r+2\dot{\theta}^2r+(\alpha-1)\frac{C}{(\alpha-1)r^\alpha}\mright]
		\end{align*}
		であるから、$r$ についての運動方程式は、
		\begin{align*}
			\frac{d}{dt}\mleft(\frac{\partial\mathscr{L}}{\partial\dot{r}}\mright)-\frac{\partial\mathscr{L}}{\partial r}&=0\\
			\frac{d}{dt}(m\dot{r})+\frac{1}{2}\mleft[2\dot{\theta}^2r+2\dot{\theta}^2r+(\alpha-1)\frac{C}{(\alpha-1)r^\alpha}\mright]&=0\\
			m\ddot{r}+\frac{1}{2}m(\alpha-1)\frac{C}{(\alpha-1)r^\alpha}&=0\\
		\end{align*}
\end{itemize}
\end{document}
