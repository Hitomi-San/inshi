%! LuaLaTeX 文書
\documentclass[report]{dennou777}

\usepackage[unicode,colorlinks=true]{hyperref}
\hypersetup{linkcolor=blue,urlcolor=teal}

\usepackage{luatexja}
\ltjsetparameter{jacharrange={-2,-3,-8}}
\usepackage[no-math,match,deluxe]{luatexja-preset}

\usepackage{textcomp}
\usepackage{luatexja-otf}

\usepackage{graphicx,xcolor}
\usepackage{pxrubrica}
\usepackage{tcolorbox}
\usepackage{indentfirst}
\usepackage{bxwareki}
\usepackage{array}

\usepackage{tikz}
\usetikzlibrary{plotmarks}

\usepackage[T1]{fontenc}
\usepackage{amsmath,mathtools,amssymb,mathrsfs,rsfso,mleftright}
\usepackage[math]{kurier}
\usepackage[euler-digits]{eulervm}
\usepackage[scaled]{beramono}
\allowdisplaybreaks[4]

\usepackage[a4paper]{geometry}

%%%%%%%%%% フォントの設定
\setsansfont[Ligatures=TeX,]{nishiki-teki}
\setsansjfont[Ligatures=TeX,]{nishiki-teki}
\setmainfont[Ligatures=TeX,]{nishiki-teki}
\setmainjfont[Ligatures=TeX,]{nishiki-teki}

%%%%%%%%%% 以下は自前コマンド

\Dtitle{北海道大学理学院 入試問題 解答例}
\Dauthor{ひとみさん}
\date{\warekitoday}

\begin{document}
\maketitle

\chapter{平成 31 年度}
\section{問題 I}

\chapter{平成 30 年度}
\section{問題 I}

\subsection{問 1}
\begin{itemize}
	\item[1-1.]
		\begin{align*}
			\mathcal{L}&=T-U\qquad\text{($T$ は運動エネルギー)}\\
			&=\frac{1}{2}m\mathbold{v}^2-U(r)\\
			&=\frac{1}{2}m\mleft[\dot{r}^2-(r\dot{\theta})^2\mright]+
				\frac{C}{\alpha-1}\frac{m}{r^{\alpha-1}}.\\
		\end{align*}
	\item[1-2.]
		$r$ についてのラグランジュの運動方程式は、
		\begin{align*}
			\frac{\partial\mathcal{L}}{\partial\dot{r}}
			&=\frac{1}{2}m\mleft(2\dot{r}\mright)\\
			&=m\dot{r},\\
			\frac{\partial\mathcal{L}}{\partial r}
			&=-\frac{1}{2}\mleft[2\dot{\theta}^2r
				+(\alpha-1)\frac{C}{(\alpha-1)r^\alpha}\mright]\\
			&=m\dot{\theta}^2r-\frac{m}{2}\frac{C}{r^\alpha}
		\end{align*}
		であるから、
		\begin{align*}
			\frac{d}{dt}\mleft(\frac{\partial\mathcal{L}}{\partial\dot{r}}\mright)
				-\frac{\partial\mathcal{L}}{\partial r}&=0\\
			m\ddot{r}-m\dot{\theta}^2r+m\frac{C}{r^\alpha}&=0\\
			\ddot{r}+Cr^{-\alpha}-\dot{\theta}^2r&=0.
		\end{align*}

		$\theta$ についてのラグランジュの運動方程式は、
		\begin{align*}
			\frac{\partial\mathcal{L}}{\partial\dot{\theta}}
			&=\frac{1}{2}m(2r^2\dot{\theta})=mr^2\dot{\theta},\\
			\frac{\partial\mathcal{L}}{\partial\theta}&=0
		\end{align*}
		であるから、
		\begin{align*}
			\frac{d}{dt}\mleft(\frac{\partial\mathcal{L}}{\partial\dot{\theta}}\mright)
				-\frac{\partial\mathcal{L}}{\partial\theta}&=0\\
			\frac{d}{dt}[mr^2\dot{\theta}]&=0.
		\end{align*}

		ここで、角運動量 $L=mr^2\dot{\theta}$ を用いると、
		$\theta$ についてのラグランジュの運動方程式は、
		\[\frac{dL}{dt}=0\]
		とかけるので、角運動量は保存する。
	\item[1-3.]
		$l=L/m=r^2\dot{\theta}$ であるから、$d\theta/dt=l/r^2$ である。
		これを用いて、$d^2/dt^2$ を計算する。

		\begin{align*}
			\frac{dr}{dt}&=\frac{dr}{d\theta}\frac{d\theta}{dt}
				=\frac{l}{r^2}\frac{dr}{d\theta}
				=l\frac{d}{d\theta}\frac{1}{r},\\
			\frac{d^2r}{dt^2}&=\frac{d}{dt}\frac{dr}{dt}\\
			&=\frac{d}{dt}\mleft[-l\frac{d}{d\theta}\frac{1}{r}\mright]\\
			&=-l\frac{d}{dt}\frac{d}{d\theta}\frac{1}{r}\\
			&=-l\mleft(\frac{d\theta}{dt}\frac{d}{d\theta}\mright)\frac{d}{d\theta}\frac{1}{r}\\
			&=-l\frac{l}{r^2}\frac{d^2}{d\theta^2}\frac{1}{r}\\
			&=-\frac{l^2}{r^2}\frac{d^2}{d\theta^2}\frac{1}{r}
		\end{align*}

		これを $r$ についてのラグランジュの運動方程式に代入すると、
		\begin{align}
			\frac{d^2r}{dt^2}+Cr^{-\alpha}-\mleft(\frac{d\theta}{dt}\mright)^2&=0\notag\\
			-\frac{l^2}{r^2}\frac{d^2}{d\theta^2}\frac{1}{r}+Cr^{-\alpha}
				-\mleft(\frac{l}{r^2}\mright)^2r&=0\notag\\
			\frac{d^2}{d\theta^2}\frac{1}{r}+C\frac{r^{2-\alpha}}{l^2}-\frac{1}{r}&=0\notag\\
			\frac{d^2}{d\theta^2}\mleft(\frac{1}{r}\mright)+
				\mleft(1-r^{3-\alpha}\frac{C}{l^2}\mright)\frac{1}{r}=0\label{30-1-1-3}
		\end{align}
		となる。
	\item[1-4.]
		$\alpha=3$ のとき、$C/l^2=\gamma, r^{-1}=R$ とおくと、
		式\eqref{30-1-1-3}は次のようにかける。
		\begin{equation}
			\frac{d^2R}{d\theta^2}-(\gamma-1)R=0\label{30-1-1-4-a}
		\end{equation}

		ここで、二次方程式 $\lambda^2-(\gamma-1)=0$ の解は、$\lambda=\pm\sqrt{\gamma-1}$
		である。

		$\gamma=1$ のとき、$\lambda=0$ となって、\eqref{30-1-1-4-a} の解は、
		$R=c_1\theta+c_2$ となる($c_1, c_2$ は任意定数)。

		$\gamma\neq1$ のとき、\eqref{30-1-1-4-a} の解は、
		$R=c_1\exp[\sqrt{\gamma-1}\,\theta]+c_2\exp[-\sqrt{\gamma-1}\,\theta]$
		となる。

		さらに、$\gamma<1$ のとき、$\sqrt{1-\gamma}=\beta$ とおいて
		さらに式を変形して、
		\begin{align*}
			R&=c_1\exp[\sqrt{\gamma-1}\,\theta]+c_2\exp[-\sqrt{\gamma-1}\,\theta]\\
			&=c_1\exp[i\beta\theta]+c_2\exp[-i\beta\theta]\\
			&=\frac{d_1-id_2}{2}\exp[i\beta\theta]+\frac{d_1+id_2}{2}\exp[-i\beta\theta]
				\qquad\text{($d_1, d_2$ は任意定数)}\\
			&=d_1\frac{e^{i\beta\theta}+e^{-i\beta\theta}}{2}
				-id_2\frac{e^{i\beta\theta}-e^{-i\beta\theta}}{2}\\
			&=d_1\frac{e^{i\beta\theta}+e^{-i\beta\theta}}{2}
				+d_2\frac{e^{i\beta\theta}-e^{-i\beta\theta}}{2i}\\
			&=d_1\cos[\beta\theta]+d_2\sin[\beta\theta]
		\end{align*}

		以上を整理して、一般解は、
		\begin{equation}
			r^{-1}=
			\begin{cases}
				c_1\cos\mleft[\sqrt{1-\dfrac{C}{l^2}}\,\theta\mright]+
					c_2\sin\mleft[\sqrt{1-\dfrac{C}{l^2}}\,\theta\mright]
					&\quad\text{$C/l^2<1$ の場合}\\
				c_1+c_2\theta
					&\quad\text{$C/l^2=1$ の場合}\\
				c_1\exp\mleft[\sqrt{\dfrac{C}{l^2}-1}\,\theta\mright]+
					c_2\exp\mleft[-\sqrt{\dfrac{C}{l^2}-1}\,\theta\mright]
					&\quad\text{$C/l^2>1$ の場合}
			\end{cases}
		\end{equation}
		である。
\end{itemize}

\subsection{問 2}
\begin{itemize}
	\item[2-1.]
		惑星に働く万有引力は $F=GmM/r^2$ である。惑星の運動の速さを $v$ と
		すると、円運動をしているので、$F=mv^2/r$ が成り立つ。

		惑星の運動エネルギーは、
		\[T=\frac{1}{2}mv^2=\frac{1}{2}rF=\frac{GmM}{2r}\]
	\item[2-2.]
		はじめの角速度を $\omega$ とすると、
		\begin{align*}
			\omega&=\sqrt{\frac{F}{mr}}\\
			&=\sqrt{\frac{1}{mr}\cdot G\frac{mM}{r^2}}\\
			&=\frac{1}{r}\sqrt{G\frac{M}{r}}.
		\end{align*}
		恒星質量が変化した後の角速度を $\omega'$ とすると、
		\begin{align*}
			\omega'&=\sqrt{\frac{F'}{m(r+dr)}}\\
			&=\sqrt{\frac{1}{m(r+dr)}\cdot G\frac{m(M+dM)}{(r+dr)^2}}\\
			&=\frac{1}{r+dr}\sqrt{G\frac{M+dM}{r+dr}}.
		\end{align*}
	
		角運動量が保存するので、
		\begin{align*}
			mr^2\omega&=m(r+dr)^2\omega'\\
			r^2\frac{1}{r}\sqrt{G\frac{M}{r}}&=(r+dr)^2\frac{1}{r+dr}\sqrt{G\frac{M+dM}{r+dr}}\\
			\sqrt{Mr}&=\sqrt{(M+dM)(r+dr)}\\
			Mr&=(M+dM)(r+dr)\\
			Mr&=Mr+rdM+Mdr\qquad\text{($dM$ と $dr$ の積を無視した)}\\
			\frac{dM}{dr}&=-\frac{M}{r}.
		\end{align*}
	\item[2-3.]
		十分ゆっくりと質量が変化し、つねに $dM/dr=-M/r$ が成り立っているので、
		$dM$ を $\Delta M=-M_0/2$ と置き換えることができる。この場合、
		\begin{align*}
			\frac{dM}{dr}\sim\frac{\Delta M}{\Delta r}&=-\frac{M_0}{r_0}\\
			\frac{-M_0/2}{\Delta r}&=-\frac{M_0}{r_0}\\
			\Delta r&=\frac{r_0}{M_0}\cdot\frac{M_0}{2}=\frac{r_0}{2}
		\end{align*}
		となって、$r$ の変分は $\Delta r=r_0/2$ である。したがって、軌道半径は
		$r_0$ の $3/2$ 倍になる。
	\item[2-4.]
		恒星質量が瞬間的に減少した場合、惑星に働く向心力が減少し、惑星が軌道の
		外側に向かって弾き飛ばされる。
\end{itemize}

\section{問題 II}

\subsection{問 1}
\begin{itemize}
	\item[1-1.]
		導体盤の面電荷密度を $\sigma$ とする。このとき、導体盤 A に直交
		して、底面積が $A$ であるような円筒面 $S$ についてガウスの法則を
		適用すると、導体盤 A から発生する電場 $E_A$ について、
		\[2AE_A=\int_S\mathbold{E}_A\cdot d\mathbold{s}=\frac{A\sigma}{\epsilon_0}\]
		が成り立つ。

		同様に、導体盤 B についても、$E_B=A\sigma/(2\epsilon_0)$で
		あり、導体盤 A, B 間に発生する電場はいずれも向きが等しいので、
		$E=E_A+E_B=\sigma/\epsilon_0$ の電場が生じる。

		このとき、コンデンサーには $Q=\sigma\pi a^2$ の電荷が帯電しており、
		A, B 間の電位差は $V=Ed$ なので、$Q=CV$ の関係から、
		\[C=\frac{Q}{V}=\frac{\sigma\pi a^2}{\sigma d/\epsilon_0}=\pi\epsilon_0\frac{a^2}{d}\]
		と求まる。
	\item[1-2.]
		コンデンサーがすでに $\pm q$ に帯電しているとき、A B 間をさらに
		微小電荷 $dq$ を移動させるのに必要な仕事は、$dW=Vdq=qdq/C$ である。
		両辺を $0$ から $Q$ まで積分すると、$W=Q^2/2C$ となるので、
		\[U=W=\frac{Q^2}{2C}\]
		である。
	\item[1-3.]
		コンデンサーが $\pm Q$ に帯電しているときに、導体盤を $\Delta r$
		だけ $z$ 方向に移動させたときに生じる静電エネルギーの差は、
		\begin{align*}
			U+\Delta U&=\frac{Q^2}{2(C+\Delta C)}\\
			&=\frac{Q^2}{2}\frac{d+\Delta r}{\epsilon_0\pi a^2}\\
			&=\frac{Q^2}{2}\frac{d}{\epsilon_0\pi a^2}\left(1+\frac{\Delta r}{d}\right)\\
			&=\frac{Q^2}{2C}\left(1+\frac{\Delta r}{d}\right)
		\end{align*}
		より、$\Delta U=(Q^2/2C)(\Delta r/d)$ である。

		一方、導体間に働く力を $F$とすると、$\Delta W=F\Delta r$ の仕事
		が必要になる。$\Delta U=\Delta W$ であるから、
		\begin{align*}
			F&=\frac{\Delta W}{\Delta r}\\
			&=\frac{\Delta U}{\Delta r}\\
			&=\frac{Q^2}{2C}\frac{\Delta r}{d}\cdot\frac{1}{\Delta r}\\
			&=\frac{Q^2}{2Cd}\\
			&=\frac{Q^2}{2\epsilon_0\pi a^2}.
		\end{align*}
	\item[1-4.]
		コイルから生じる起電力は、$V_{\mathrm{L}}=-L\,dI/dt$ である。一方、
		コンデンサーから生じる起電力は、$V_{\mathrm{C}}=Q/C$ である。回路
		が閉じているので、それぞれの起電力は等しくなる。
		\begin{align*}
			V_{\mathrm{C}}&=V_{\mathrm{L}}\\
			\frac{Q}{C}&=-L\frac{dI}{dt}\\
			\frac{Q}{C}&=-L\frac{d^2Q}{dt^2}\\
			LC\frac{d^2Q}{dt^2}+Q&=0.\qquad\text{($I=dQ/dt$ の関係を用いた)}
		\end{align*}

		$Q=e^{\lambda t}$ とおいて、上の微分方程式を解く。
		\begin{align*}
			LC\frac{d^2}{dt^2}e^{\lambda t}+e^{\lambda t}&=0\\
			LC\lambda^2e^{\lambda t}+e^{\lambda t}&=0\\
			\lambda&=\sqrt{\frac{1}{LC}}i
		\end{align*}
		ゆえに、一般解は$Q[t]=c_1\exp[\sqrt{1/LC}\,it]+c_2\exp[-\sqrt{1/LC}\,it]$
		である。初期条件より、$Q[0]=Q, I[0]=dQ/dt[0]=0$ であるから、
		$c_1=c_2=Q/2$。ゆえに、
		\begin{align*}
			Q[t]&=\frac{Q}{2}\exp\left[\sqrt{\frac{1}{LC}}\,it\right]
				+\frac{Q}{2}\exp\left[-\sqrt{\frac{1}{LC}}\,it\right]\\
			&=Q\cos\left[\frac{t}{\sqrt{LC}}\right]\\
			I[t]=&-\frac{Q}{\sqrt{LC}}\sin\left[\frac{t}{\sqrt{LC}}\right]
		\end{align*}
\end{itemize}
\end{document}
